\documentclass[a4paper,11pt]{article}
\usepackage[utf8]{inputenc}

%opening
\title{CFD Lab Project: Simulation of the MI Building}
\author{Group 2: M. Andreev, G. Chourdakis, I. Tominec \\ Advisor: Kaveh Rahnema}

\begin{document}

\maketitle

\section{Problem definition}
In this project, we are trying to mix arbirary geometries with parallel processing
in our Lattice-Boltzmann Method implementation.
As a study case, we choose a geometry that looks like a 2D representation (top view) of the
TUM Mathematics-Informatics building. In order to be able to apply the Lattice-Boltzmann Method
in this study case (and get usuable results), we need a very fine grid, so we need
to have a parallel code.

\section{Challenges}
The main challenge of the project is to parallely process partitions of different
sizes, that need to communicate only in specific points. We could just mix the
already known approaches, i.e. a big rectangle including all the geometry, distributed
evenly to all the processors. This way, we would spend a lot of computational effort
in order to process big areas of ``inactive'' boundary cells (e.g. between the
``wings'' of the MI building). Another challenge is to balance the work each processor
has to do, i.e. each processor needs to be assigned multiple partitions, as the partitions
may have different sizes.

Other challenges, related to the study case, are to define the geometry,
create geometry input files, define boundary conditions and flow parameters.

\section{Approach}
First of all, we define out study case, in order to choose a computational approach.
The building is modelled as a 2D top-view cross-section, near the floor and it
is represented as several rectangles. The rectangles are dimensionless and scalable
according to a size parameter ``x''. The physical dimensions of the building were
defined by combining in-place measurements and the sizes of a ``fire plan'' of
the building. Average dimensions were chosen, e.g. for the main hall length=152m (or 32x)
and width=19m (or 4x), keeping a constant ratio of 8:1. The coordinate system
is chosen as x:long side of the main hall (length) and y:short side (width).
The origin is defined near the main entrance, so that it belongs to a big rectangle
that contains the whole building.

Some rectangular areas in the interior are assigned a no-slip boundary condition,
in order to represent tables in the main hall. Five doors of the main hall are
simulated, all open, either as inflows (with constant velocity), or as outflows:
northern (main entrance), eastern (to LRZ), next to cantine, southern and western (next to the library).

The geometry is divided in 14 partitions, numbered $\{0\dots13\}$. The main hall is
divided in 4 rectangular partitions, connected in the x-direction. Each of the 10
``wings'' is a partition on itself. Each partition is defined by a different geometry (pgm)
file. In order to produce the pgm files, a Matlab script was developped, that takes
as main parameter the size parameter ``x'' (lattice cells per x unit). Each file (containing the flagfield
for the 2D domain) is loaded by the corresponding process and a result vtk file for
each partition is produced.

Each partition uses a ``parallel boundary'' flag at the areas where it has to communicate.
Every cpu know the its neighbors that it has to communicate with. Each partition
is stored in a different array. Each processor also knows which parts of each partition
it has to communicate with the connected partitions.

It should be noted that, although the main developement is complete,
some problems exist in the results. It is in debugging state, but we continue trying to
fix the problems.


\end{document}
